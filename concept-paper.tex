\documentclass[conference]{IEEEtran}
\IEEEoverridecommandlockouts
% The preceding line is only needed to identify funding in the first footnote. If that is unneeded, please comment it out.
\usepackage{cite}
\usepackage{amsmath,amssymb,amsfonts}
\usepackage{algorithmic}
\usepackage{graphicx}
\usepackage{textcomp}
\usepackage{xcolor}
\def\BibTeX{{\rm B\kern-.05em{\sc i\kern-.025em b}\kern-.08em
    T\kern-.1667em\lower.7ex\hbox{E}\kern-.125emX}}
\begin{document}

\title{KEY PERFORMANCE INDICATOR CALCULATOR FOR CALLS IN CUSTOMER SUPPORT CALL CENTER IN UGANDAN FINTECHS\\}

\author{\IEEEauthorblockN{1\textsuperscript{st} Joseph Lusoma}
\IEEEauthorblockA{\textit{master of science in computer science} \\
\textit{Makerere University}\\
Kampala, Uganda \\
joseph.lusoma@students.mak.ac.ug}
\and
\IEEEauthorblockN{2\textsuperscript{nd} Marvin Sendikaddiwa}
\IEEEauthorblockA{\textit{master of science in computer science} \\
\textit{Makerere university}\\
Kampala, Uganda \\
marvin.sendikaddiwa14@student.mak.ac.ug}
}

\maketitle

\begin{abstract}
The Key Performance Indicator for Customer Support call centers is a machine learning project that involves building a system and a model that takes in recorded voice calls as input, translates conversation to text using speech-to-text recognition, and process the text conversation to pick measurable insights such as Turn Around Time (TAT), Service Level Agreement (SLA) and other useful details in form of graphs or tabular aggregations which is easily interpreted by the company management and senior team to make more informed decisions.
\end{abstract}

\begin{IEEEkeywords}
Key Performance Indicator(KPI), Turn Around Time (TAT), Service Level Agreement (SLA), speech-to-text
\end{IEEEkeywords}

\section{Introduction}
Reduced costs in telecommunications and information technology have made it increasingly cost-effective to consolidate information delivery functions such as Voice over Internet Protocol (VoIP), leading to the creation of groups that specialize in handling customer telephone calls. For many of these groups, their primary function is to receive customer-initiated phone calls. However, the data that is processed during voice calls is not analysed and evaluated, which limits organizations in evaluating their performance and effectiveness.

\section{Background}
The fastest and easiest way to understand your customer base and their problems is to reach out to them. However, there are problems associated with this, such as time-consuming, the need for many trained workers, and practically it is done between a defined period.

Most customer-facing companies, such as fin-techs, at least have a complete customer experience or subcontract to external companies that provide both operational and technical support on behalf of their business.

The above customer support mods have a similar issue. Some of these issues included but were not limited to: - Difficult to gain insight into recurring customer issues, business performance, employee recognition by customers, SLAs and other potential business areas requested by potential customers. Therefore, there is a need to incorporate artificial intelligence into KPI measurement systems.

\section{Significance of Project}

\subsection{Customer Support Call Center}
Customer service call centers are the first point of contact between customers and businesses. These centers are responsible for answering customer questions, solving problems and providing information about products and services. In order to provide excellent customer service, companies need to have a system that accurately identifies and categorizes calls. A good call center management software should be able to identify the type of call and assign it to the appropriate agent based on their skills.

\subsection{Key Performance Indicator (KPI)}
A KPI is a quantifiable measure of performance that is used to evaluate the success of a business or organization. KPIs are often used to measure the effectiveness of marketing campaigns, sales strategies, and operational processes. Companies use KPIs to determine whether they are meeting their goals and objectives.

\subsection{Machine Learning and Artificial Intelligence}
Machine learning is a subfield of artificial intelligence that focuses on algorithms that teach computers how to learn without being explicitly programmed. Machine learning involves using statistical techniques to create computer programs that automatically improve over time.

\section{Project Scope}
\begin{itemize}
\item To develop a desktop application that uses machine learning to analyse and categorizes VoIP calls according to the organization’s activities in relation to the issues being received through calls as well as making futuristic performance predictions.
\item Developing a model using Machine Learning (AI) techniques to predict KPIs based on historical data.
\item Collect data from various organizations that currently use VoIP systems in their work areas.
\item Evaluating the performance of the model using real world data.
\item Identifying the best performing model among different models developed.
\item Analysing the results obtained from the evaluation phase and identifying the reasons behind the poor performance of some models.
\item Providing recommendations to improve the performance of the model.
\item Presenting the findings at the end of the project.
\item Preparing a report about the project.
\end{itemize}

\section{Problem Statement}
The problem this project will address is the un-analysed voice calls received at the call centers. Performance evaluation of the organization tends to be hectic as there are usually large piles of data to evaluate in limited time. This gives a lot of room for making mistakes during evaluation.

Call center operators, put more emphasis on solving issues that come from calls but none on the number of calls received concerning a particular issue so that they can get long term solutions to the problem.

This project is designed to provide accurate information about calls by providing a summarized and categorized output of received calls in form of a simple graph or table. The project is aimed at increasing the accuracy of the information about voice calls. The final product will be a desktop that will analyze customer calls thereafter out puts the results in form of a KPI graph.


\section{Research Questions}
Are Key Performance Indicators metrics collected efficiently for customer support personnel that handle customer calls?

Is the Turn Around Time for calls measured with a reasonable degree of accuracy by Customer Support Managers?

Is Issue logging and tracking at customer support calls the most effective way to get measurable and discrete insights from customers?

\section{Research Objectives}
\begin{itemize}
\item To create an effective speech analysis system for phone call conversations in business customer support centers.
\item To easily track the Service Level Agreements (SLAs) of phone calls made in business customer support centers.
\item To get the average Turn Around time (TAT) between issue reception to issue resolution in business customer support centers.
\item To accurately track Key Performance Indicators (KPI) for customer support personnel when they respond to customers in call centers.
\end{itemize}

\section{Dataset Description}

\section{Artificial Intelligence Methodology}
The system will be able to receive calls and the voice analysis algorithm will perform real time voice analysis. At this level, accurate information is of a high importance since it determines how calls will be correctly categorized to the right service delivery in relation to the client’s issue. The algorithm will use a combination of Speech recognition APIs from well-known portable languages such as Java and Python. The algorithm will be able to recognize speech through employing techniques such as:-
\begin{itemize}
\item Simple pattern matching where each spoken word is recognized entirely
This kind of speech recognition is usually employed in the automated call center and has been answered by a computerized switchboard.
\item 	Pattern and feature analysis where each word is broken into bits and recognized from key features such as the vowel it contains
\item Language modeling and feature analysis, in which the knowledge of the grammar and probability of certain words and sounds following on from one another is used to speed up recognition and accuracy.
\end{itemize}
The algorithm will convert speech to text and then make comparison of this text to the inbuilt category features.

The category features will include the organizations service and products for instance for the case of a telecommunication company. Issues like airtime, sms, internet usage, mobile money, inquiries about service delivery and network signals will be included in the category features.

\begin{thebibliography}{00}
\bibitem{b1} Shon, Suwon \& Brusco, Pablo \& Pan, Jing \& Han, Kyu \& Watanabe, Shinji. (2021). Leveraging Pre-Trained Language Model for Speech Sentiment Analysis. 3420-3424. 10.21437/Interspeech.2021-1723.
\bibitem{b2} Petkovic, D., Kobzik, L., \& Re, C. (2018). Machine learning and deep analytics for biocomputing: call for better explainability. In PACIFIC SYMPOSIUM ON BIOCOMPUTING 2018: Proceedings of the Pacific Symposium (pp. 623-627).
\bibitem{b3} Sudarsan, V., \& Kumar, G. (2019). Voice call analytics using natural language processing. Int. J. Stat. Appl. Math, 4, 133-136.
\bibitem{b4} Xu, Y. (2022). English speech recognition and evaluation of pronunciation quality using deep learning. Mobile Information Systems, 2022.
\bibitem{b5} Koops, S., Brederoo, S. G., de Boer, J. N., Nadema, F. G., Voppel, A. E., \& Sommer, I. E. (2022). Speech as a Biomarker for Depression. CNS \& Neurological Disorders-Drug Targets (Formerly Current Drug Targets-CNS \& Neurological Disorders).
\bibitem{b6} Raina, V., \& Krishnamurthy, S. (2022). Natural language processing. In Building an Effective Data Science Practice (pp. 63-73). Apress, Berkeley, CA.
\bibitem{b7} Rathor, S., \& Agrawal, S. (2022). Sense understanding of text conversation using temporal convolution neural network. Multimedia Tools and Applications, 81(7), 9897-9914.
\bibitem{b8} Rathor, S., \& Agrawal, S. (2022). Sense understanding of text conversation using temporal convolution neural network. Multimedia Tools and Applications, 81(7), 9897-9914.
\bibitem{b9} Liu, T., Meyerhoff, J., Eichstaedt, J. C., Karr, C. J., Kaiser, S. M., Kording, K. P., ... \& Ungar, L. H. (2022). The relationship between text message sentiment and self-reported depression. Journal of affective disorders, 302, 7-14.
\bibitem{b10} Brunello, A., Marzano, E., Montanari, A., \& Sciavicco, G. (2022). A combined approach to the analysis of speech conversations in a contact center domain. arXiv preprint arXiv:2203.06396.
\bibitem{b11} 11.	Park, T. J., Kanda, N., Dimitriadis, D., Han, K. J., Watanabe, S., \& Narayanan, S. (2022). A review of speaker diarization: Recent advances with deep learning. Computer Speech \& Language, 72, 101317.
\bibitem{b12} Umair, M., Mertens, J. B., Albert, S., \& de Ruiter, J. P. (2022). GailBot: An automatic transcription system for Conversation Analysis. Dialogue \& Discourse, 13(1), 63-95.
\end{thebibliography}
\vspace{12pt}
\end{document}




